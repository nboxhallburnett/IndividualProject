\section{Application of VR in Non-Euclidean worlds}
\label{lr:cross}

\begin{multicols*}{2}
	\subsection{Introduction}
	\label{lr:cross:intro}
%		In this section I will introduce the concepts to be covered in this section, namely how one can apply the key concepts for perception and presence in VR to a game world using non-standard geometry.
	
	\subsection{Application}
	\label{lr:cross:application}
%		Here I will be covering any existing research which relates to the combination of VR and non-Euclidean geometry.
		(PLACEHOLDER) The following sources will be discussed in this section:
		\begin{itemize}
			\item \cite{Cruz-Neira1993} - Surround-screen projection-based virtual reality
			\item As referenced in \ref{lr:vr:perception}, \cite{Wright2014} - Using virtual reality to augment perception, enhance sensorimotor adaptation, and change our minds - Discuss how it could also relate to sensorimotor adaptations allowing a user to get accustomed to existing i\textsl{}n a non-Euclidean world.
		\end{itemize}
	
	\subsection{Tools \& Techniques}
	\label{lr:cross:tools}
%		This section will cover tools which are applicable for the creation of VR/Non-Euclidean virtual environments.
		(PLACEHOLDER) In this section I will be covering the various tools I could use to create the proposed system, these tools and techniques will include:
		\begin{itemize}
			\item Software 
			\begin{itemize}
				\item Custom engine
				\begin{itemize}
					\item Oculus SDK 0.8
					\item NVIDIA Gameworks VR
					\item AMD LiquidVR
					\item DirectX 11.2 / 12
					\item OpenGL 4.5
				\end{itemize}
				\item Modify existing engine
				\begin{itemize}
					\item Unity 5
					\item Unreal 4
				\end{itemize}
			\end{itemize}
			\item Hardware
			\begin{itemize}
				\item Head-Mounted Display
				\begin{itemize}
					\item Oculus Rift DK2
					\item HTC Vive
				\end{itemize}
				\item Input
				\begin{itemize}
					\item Keyboard/Mouse
					\item Game controller (PS4, XBox One, etc.)
					\item Web-cam/Kinect
				\end{itemize}
			\end{itemize}
			\item Project Management
			\begin{itemize}
				\item Time Planning
				\begin{itemize}
					\item Agile
					\item Waterfall
				\end{itemize}
				\item Source Control
				\begin{itemize}
					\item Git
					\item SVN
				\end{itemize}
			\end{itemize}
		\end{itemize}
		Resources for reference:
		\cite{Bruce2012} - Custom engine v modify an existing one - debate. Compares the pros and cons of both modifying an existing engine and creating a custom solution. ('Leaky Abstractions' with an existing engine, balance engine efficiency and actual system development with a custom engine, etc.).
		
	\subsection{Conclusion}
	\label{lr:cross:conclusion}
%		This section will sum up the points discussed in the above subsections.
		(PLACEHOLDER) Sum up the points discussed in the above sections. Example: current implementations of non-euclidean geometry in released games have always directly used, or modified an existing generic game engine to create the VE, and are therefore not optimised to be used in a non-euclidean world.
	
\end{multicols*}
