\documentclass[abstract=on,12pt]{scrreprt}

% Set packages to be used
\usepackage[hidelinks]{hyperref}	% Adds support for hyperlinks
\usepackage[british]{babel}			% Set the language to UK English
\usepackage[a4paper]{geometry}		% Set the paper type
\usepackage[T1]{fontenc}			% Change the encoding to allow for more character types
\usepackage{booktabs}				% More professional looking table layouts
\usepackage{etoolbox}				% Lets you do fancy stuff with section types
\usepackage{fancyhdr}				% Fancy Headers
\usepackage{graphicx} 				% For loading graphic files
\usepackage{multicol}				% Used to set column counts
\usepackage{pdfpages}				% Used to include external PDFs in the document
\usepackage{ragged2e}				% for '\RaggedRight' macro (allows hyphenation)
\usepackage{setspace}				% Sets the spacing between lines
\usepackage{tabularx}				% for 'tabularx' environment and 'X' column type
\usepackage{titlesec}				% Allows custom formatting for sections
\usepackage{apalike}				% Format references in APA styling
\usepackage{gensymb}				% Allow inclusion of special characters, such as degrees
\usepackage{lmodern} 				% Type1-font for non-english texts and characters

% Set document specific info
\title{Is Immersion Possible in Non-Euclidean Virtual Environments?}
\subtitle{CHP2524 - Individual Project Report}
\author{Nathan Boxhall-Burnett}
\setlength{\columnsep}{1cm}			% Set the spacing between columns
\onehalfspacing						% Set the spacing between the lines of text
\graphicspath{Images/}				% Set the default path to use for images

% Set the page number and running title as header
\pagestyle{fancy}
\fancyhf{}
\fancyheadoffset{0cm}
\renewcommand{\headrulewidth}{0pt}
\renewcommand{\footrulewidth}{0pt}
\fancyhead[R]{\thepage}
\fancyhead[L]{Is Immersion Possible in Non-Euclidean Virtual Environments?}
\fancypagestyle{plain}{%
	\fancyhf{}%
	\fancyhead[R]{\thepage}%
	\fancyhead[L]{Is Immersion Possible in Non-Euclidean Virtual Environments?}
}

% Set formatting for chapter titles
\titleformat{\chapter}[hang]{\huge\bfseries}{\thechapter}{1em}{\huge}

% Set formatting for section titles
\titleformat{\section}[hang]{\large\bfseries}{\thesection}{1em}{\large}

% Set formatting for section titles
\titleformat{\subsection}[hang]{\normalsize\bfseries}{\thesubsection}{1em}{\normalsize}

% Add the abstract to the table of contents
\patchcmd{\abstract}{\titlepage}{\titlepage
	\addcontentsline{toc}{chapter}{Abstract}}{}{}

% Begin the document
\begin{document}

	% Title Page
	\maketitle

	% Set the inital pages to use roman numerals for numbering
	\pagenumbering{roman}

	% Table of contents
	\tableofcontents

	% Abstract
	\begin{abstract}
		\thispagestyle{plain}
		expression of general problem/area

		specific problem/objective attempted

		what you have achieved

		lessons learned

		%% DO NOT START BY SAYING WHAT YOU HAVE DONE 
		%% DO NOT INCLUDE JARGON / DETAILS 
		% Eg bad start “I have implemented a C\# database running on Linux v4.5.2 … “
		% Eg bad start “ I have created a web application” – start with overview and problem area first.
		%% MAKE IT UNDERSTANDABLE TO ALL ! 


%		With Virtual Reality (VR) development and availability increasing as rapidly as it has in recent years, especially in the video game market, experiences that are capable from Virtual Environments (VE) that are not limited to the restrictions of the real world, and the effectiveness they have in a users perception, need to be considered.
%		This review explores the existing literature that covers the various requirements for creating a VE that consumers would perceive as 'realistic' in VR. As well as this, it also covers how these requirements have been, and can, be adapted to work inside of a world which does not follow the restrictions of real world geometry, such as in non-Euclidean or Escheresque space.
%		Gaps in the current literature are outlined, alongside perspective areas for future research.
	\end{abstract}

	% Set formatting for section titles
	\titleformat{\chapter}[hang]{\large\bfseries}{\thechapter}{1em}{\large}

	% Remove section numbering for the acknowledgements
	\setcounter{secnumdepth}{-2}

	% Document Acknowledgements
	\chapter{Acknowledgements}
		I would like to thank my supervisor Hugh Osborne, and examiner Duke Gledhill, for their invaluable feedback and suggestions throughout.
		I would also like to thank all the participants who took the time to take part in the product experiments.

	% Set formatting for chapter titles
	\titleformat{\chapter}[hang]{\huge\bfseries}{\thechapter}{1em}{\huge}

	% Reset the page count, reset the numbering style to arabic, and add section numbering again
	\newpage
	\renewcommand\thepage{\arabic{page}}
	\pagenumbering{arabic}
	\setcounter{secnumdepth}{3}

	% Introduction
	\chapter{Introduction}
\label{intro}

Here I will be introducing the project and the various sections of the report document in detail, making sure to cover the terms and concepts that will be discussed to ensure the reader has full understanding of the report contents.

	% Literature Review
	\chapter{Literature Review}
\label{lr}

	\section{Introduction}
	\label{lr:intro}

		This review explores the existing literature that covers the various requirements for creating a VE that consumers would perceive as \enquote{realistic} in VR.
		As well as this, it also covers how these requirements have been, and can be, adapted to work inside of a world which does not follow the restrictions of real world geometry, such as in non-Euclidean or Escheresque space.
		Gaps in the current literature are outlined, alongside prospective areas for future research.

		There are two main areas which make up the requirements for the aims of this study.
		The first area is identifying the components which are required to achieve a sense of presence in a Virtual Environment (VE) when in VR, and how they influence a users feeling of immersion within it.
		The second area covers research into the uses and effects of non-Euclidean geometry in virtual environments. % I did start adding something here, if you remember what it was then have at it

		The following sections in this chapter cover both the aforementioned areas, as well as research into the tools and techniques which can be used to build the proposed system, and a conclusion which summarises the reviewed literature and its impacts on the proposed project.

	\section{Presence in Virtual Reality}
	\label{lr:vr}

		Presence is the term used in VR systems to describe a user's sense of existing inside a VE. Because of this it can be seen as one of, if not the most important feature of the design for a VR system, as well as a precursor for a user becoming fully immersed in the VE.

		There are a few main features which are key to achieving a sense of presence in a VE. One example would be the general perception of the user for things such as depth perception, the ability for a person to calculate the distance of a point from themselves, and sensorimotor adaptation, which is the calibration of a user's senses to fit their environment.
		As well as this, there are specified areas such as affordance detection, the ability for a person to calculate interaction with an object or environment, which helps assist a user with navigation and interaction in a VE.
		The following sections will review existing literature on the aforementioned features.

		\subsection{Affordances in VR}
		\label{lr:vr:affordances}
			Affordance is especially important to get right in VR, particularly in a video game environment, due to the world a user is in being designed to be interacted with.
			Unlike what common sense would dictate, research on affordance in the real world may not be as applicable for when designing interactions in a virtual world as expected.
			A study \cite{Regia-Corte2012} focusing on a person's perception for whether or not they could stand upright on an object at various angles in a virtual world found that, contrary to expected results, users would tend to be more cautious when judging their ability in a virtual environment (critical angle 21.98\degree) compared to an equivalent object in the real world (critical angle 30\degree), even with a lack of apparent risk.

			This study covered the various requirements for the experiments well, however there were a few areas in which it could still be expanded upon.
			These areas include comparing results from using a variety of VR systems to judge responses based on hardware response times, display resolutions, and refresh rates, as well as increases and decreases in model and texture quality (not just the texture itself as they did) for judging the surfaces of the objects.

			Another way in which affordance directly affects a user's sense of presence in VR is the simulation of real locomotion for the user.
			In a study \cite{Turchet2015} covering the ways in which a user's movement is recorded and displayed in a VE (including, but not limited to, individual foot tracking, arm, leg, and head tracking, movement speed tracking, etc.), results found that while an increase in the coverage of the input also increased the sense of presence for the user, it was more important to focus the areas based on their relevance to the scenario at hand.
			As well as this, the study found that while the input tracking itself was important, to attain a greater sense of presence it was also required for the physical representation of the user to match their real physical selves as much as possible, such as gender, height, and weight.

			This study, while extremely thorough in its coverage of the various possible elements required to generate a sense of presence, could have recorded and displayed its results in a more usable manner than it did.
			Due to the way the experiments were conducted, the results do not indicate any sense of importance for each individual step in its contribution to the feeling of presence in the user.
			Because of this, further research could be undertaken to create a weighted system designed around the experiments conducted, signifying its importance for specific VR systems, creating a more convenient set of data for future use.

		\subsection{Perception in Virtual Environments}
		\label{lr:vr:perception}
			Perception is a key feature to get right in VR, as incorrect calibration for features such as head tracking and input response can have both unintended, and undesired side-effects. In studies \cite{Wright2006}  \cite{Wright2009} \cite{Wright2011} \cite{Wright2013} \cite{Wright2014} around the effects of perception augmentation and sensorimotor adaptations, results showed that tests around sensorimotor processes that made use of Virtual Reality systems could have both intended and unintended effects on the participants central nervous system, both in the short term and the long term.
			As well as this, the effects of sensorimotor adaptations were shown to not only effect the user during the exposure of the VE, but in prolonged exposure (<5 minutes), the adaptations made during the exposure could last for >2 minutes after leaving the VE \cite{Wright2013}.

			These studies thoroughly cover the short-term effects that a wide variety of virtual and physical stimulation has on perception in a VE, and present their findings in a concise and readable manner.
			However, other than passing mentions about the possibilities of them existing, the studies did not cover or follow up on any longer-term effects that the exposure of the experiments has on participants.

			In a separate study \cite{Akizuki2005}, the effects of visual and physical stimulation in VR on a participants postural control was also covered.
			This study found that, unlike in Wright's experiments, the effects that delayed visual responses from a VR HMD relative to a participants physical movements did not have any prolonged effects on a participants sensorimotor control outside of the VE.

			The varied results from Akizuki's study compared to Wright's could be attributed to the quality of the hardware used for the VR headsets, with Akizuki's experiments using hardware which is upwards of 9 years older than that of Wright's, the participants in the experiments may have had a higher sense of immersion in Wright's experiments, allowing a greater possibility for sensorimotor adaptation to the VE.

		\subsection{Summary}
		\label{lr:vr:conclusion}
			Due to recent relevant technology being both of a high quality, and low enough cost to warrant consumer interest, as well as its applications in not just Video Games, but in Medicine, Psychology, and Education, Virtual Reality is a very highly researched area.

			Studies have been conducted to cover an extremely wide range of applications and effects for VR, and because of this, there is a very solid foundation of work which can be both applied for practical uses, as well as a platform to build upon for future research.

			This is especially true in the two main focus points for the literature covered in this section, and will be beneficial for reference during the creation of the proposed system.


	\section{Non-Euclidean Geometry in Virtual Environments}
	\label{lr:ne}

		Attempts to find existing literature which covers the use or effects of non-Euclidean geometry in virtual environments, with or without the use of VR, concluded that there is a distinct lack of academic research surrounding the subject area.
		Research does exist in the mathematics behind non-Euclidean geometry \cite{Maric2014} \cite{Turner2009}, however this is not applicable to this study as it surround the theoretical practices of the uses, and not related to practical applications for implementations in virtual environments.

		The lack of research in the subject area could be due to the fact that research into the effects of VR is a relatively new subject area, so the more niche areas are less likely to have been studied at this point.
		This does, however, open up a broad area for the research to be conducted by the proposed project, as well as opportunities for further research beyond it.

	\section{Tools and Techniques}
	\label{lr:tools}

		% This section will cover tools which are applicable for the creation of VR/Non-Euclidean virtual environments.
		% TODO: In this section I will be covering the various tools I could use to create the proposed system, these tools and techniques will include:
		\begin{itemize}
			\item Software
			\begin{itemize}
				\item Custom engine
				\begin{itemize}
					\item Oculus SDK 0.8
					\item NVIDIA Gameworks VR
					\item AMD LiquidVR
					\item DirectX 11.2 / 12
					\item OpenGL 4.5
				\end{itemize}
				\item Modify existing engine
				\begin{itemize}
					\item Unity 5
					\item Unreal 4
				\end{itemize}
			\end{itemize}
			\item Hardware
			\begin{itemize}
				\item Head-Mounted Display
				\begin{itemize}
					\item Oculus Rift DK2
					\item HTC Vive
				\end{itemize}
				\item Input
				\begin{itemize}
					\item Keyboard/Mouse
					\item Game controller (PS4, XBox One, etc.)
					\item Web-cam/Kinect
				\end{itemize}
			\end{itemize}
			\item Project Management
			\begin{itemize}
				\item Time Planning
				\begin{itemize}
					\item Agile
					\item Waterfall
				\end{itemize}
				\item Source Control
				\begin{itemize}
					\item Git
					\item SVN
				\end{itemize}
			\end{itemize}
		\end{itemize}
		Resources for reference:
		\cite{Bruce2012} - Custom engine v modify an existing one - debate. Compares the pros and cons of both modifying an existing engine and creating a custom solution. ('Leaky Abstractions' with an existing engine, balance engine efficiency and actual system development with a custom engine, etc.).

	\section{Conclusion}
	\label{lr:conclusion}

		In Virtual Reality, non-standard geometry is definitely an area with potential to engage, and possibly spark an interest within a user about the capabilities and uses of such concepts.

		% TODO - Reword this?
		This review has given an overview into the various components required for the creation of an environment suitable for use in VR, how non-standard geometrical worlds can be created for use in a VE, and how the areas both require adaptation in order to work together.

		In terms of VR, existing literature is very thorough in its coverage of the possibilities and effects of its use both within and outside of video game environments.
		Several key areas that should be considered when designing a virtual environment for VR were outlined such as affordance detection and potential sensorimotor adaptation, and how they can effect a users sense of presence and immersion.
		Although there is definitely still room for further research into the more niche areas of study, there is a solid foundation for use as a reference for further development in the field.

		For non-Euclidean geometry, however, there is a distinct lack of academic research into its use within virtual environments, with most studies focusing instead on the theoretical applications for it.
		Reports around the use of non-standard geometry in video games does exist, however due to the non-scholarly nature of the sources, which are mainly video game journalist articles or developer interviews, they cannot be completely trusted for academic purposes.

		Due to combination of the findings surrounding these areas, research into the use of non-Euclidean geometry within a VR system is very open to exploration, and as such, research would be beneficial as a possible basis for further study, perhaps even encouraging it.

	% Maybe other background stuff too, who knows

	% Models/Product Design stuff
	\chapter[Product]{Design and Implementation}
\label{design}

\begin{multicols*}{2}

	\section{Introduction}
		% TODO - Here I will give an overview of the following sections.

	\section{Tools}
		% TODO - This section will cover the various tools I used to complete the product itself, and why I chose them as opposed to any alternatives.

		% Talk about how you decide to use an existing engine you knew well (save time reinventing the wheel for things like lighting, camera stuff, etc)
		% Also you wanted to limit the amount of outside variables as possible, so an existing engine helped with that

	\section{Product model}
		% TODO - This section will contain models of the various states and relations between the functions and other workings of the product, for areas such as the transition between non-Euclidean world areas.
		\autoref{appendix:code:player}

\end{multicols*}

	\begin{figure}[h]
		\label{design:fig:maths}
		\includegraphics[width=0.8\textwidth]{Images/Position}
		\centering
		\caption{2D representation of the calculations for the position and rotation of cameras.
			See \autoref{appendix:code:camera} for the implementation}
	\end{figure}

	\begin{figure}[h]
		\label{design:fig:scene}
		\includegraphics[width=1\textwidth]{Images/Lines_Everywhere2}
		\centering
		\caption{Example view of a scene.
			Red lines are connections between points,
			yellow lines are connections visible to the player,
			and green lines are the direction the connectors are facing}
	\end{figure}

	\begin{figure}[h]
		\label{design:fig:game}
		\includegraphics[width=1\textwidth]{Images/NE_View}
		\centering
		\caption{Example of view from inside the Non-Euclidean scene, displaying an area which is larger on the inside}
	\end{figure}

\begin{multicols*}{2}

	\section][Environment Design]{Design of experiment environments}
		% TODO - Here I will be covering the designs of the various areas of the virtual environments to be used in the experiments themselves, and why they are applicable for use for the tests.


\end{multicols*}


	% Experiment
	\chapter{Experiment}
\label{exp}

	\section{Introduction}
	\label{exp:intro}

		As a way to get the widest possible range of data from the scenes created, two different experiments were planned (\enquote{Experiment 1}, and \enquote{Experiment 2}).
		The experiments were set up in a way that a participant in an experiment would either be shown the standard or non-Euclidean game scene first, and once they had successfully navigated around it, they would then be shown the other.
		\enquote{Experiment 1} had the participants viewing the standard scene first and the non-Euclidean scene second, with \enquote{Experiment 2} having the participants view the non-Euclidean scene first and the standard scene second.

		A total of 10 participants took part in the experiments, with 5 participants taking part in \enquote{Experiment 1}, and the other 5 \enquote{Experiment 2}.
		Having the participants view the two scenes in different orders allowed for the results of the experiments to not only cover the effects of the respective scenes geometry, but also to see if being exposed to the other environment in any way impacts their immersion to the one they are currently in. % TODO: Reword this

		During the experiments, participants were asked to fill out a 2 page questionnaire, which can be seen in \autoref{appendix:question}.
		After viewing the first scene, participants were asked to fill out the first page of the questionnaire. Once they'd filled the page, they would view the second scene, after which they were asked to fill out the second page. % TODO: Reword this

		The analysis of the gathered results will be covered in three sections. The first will cover the results from all 10 experiments solely on the data gathered for the standard scene; The second covers the results from all 10 experiments solely on the data from the non-Euclidean scene; And finally the third covers the comparison between the two scene types depending on the order of viewing. % TODO: Reword this?

		% REMEMBER TO SPEAK ABOUT THE MATHS

	\section{Experiments}
	\label{exp:exp}

		\subsection{Standard Euclidean Geometry}
		\label{exp:exp:standard}

			% Talk about the results specific to the standard euclidean geometery here.
			% Used to get a base result for immersion
			% Consistent high ratings for immersion
			% Navigation comfort was varied, but talk about the high mean values

			\begin{figure}[H]
				\includegraphics[width=0.7\textwidth]{Images/Standard_Immersion}
				\centering
				\caption{Immersion rating in standard geometry test scene}
				\label{exp:fig:standard_immersion}
			\end{figure}

			\begin{figure}[H]
				\includegraphics[width=0.7\textwidth]{Images/Standard_Comfort}
				\centering
				\caption{Navigation Comfort rating in standard geometry test scene}
				\label{exp:fig:standard_comfort}
			\end{figure}

			\begin{figure}[H]
				\includegraphics[width=0.7\textwidth]{Images/Standard_Relation}
				\centering
				\caption{Relation between sense of immersion and navigation comfort in standard geometry test scene}
				\label{exp:fig:standard_relation}
			\end{figure}

		\subsection{Non-Euclidean Geometry}
		\label{exp:exp:ne}

			% Talk about the results specific to the non-eucldean geometry here
			% More varied results
			% Same Mean
			% More responses in the higher end of the scale - talk about the written feedback
			% Lower comfort with navigation

			\begin{figure}[H]
				\includegraphics[width=0.7\textwidth]{Images/NE_Immersion}
				\centering
				\caption{Immersion rating in Non-Euclidean geometry test scene}
				\label{exp:fig:ne_immersion}
			\end{figure}

			\begin{figure}[H]
				\includegraphics[width=0.7\textwidth]{Images/NE_Comfort}
				\centering
				\caption{Navigation Comfort rating in Non-Euclidean geometry test scene}
				\label{exp:fig:ne_comfort}
			\end{figure}

			\begin{figure}[H]
				\includegraphics[width=0.7\textwidth]{Images/NE_Relation}
				\centering
				\caption{Relation between sense of immersion and navigation comfort in Non-Euclidean geometry test scene}
				\label{exp:fig:ne_relation}
			\end{figure}

		\subsection{Comparisons}
		\label{exp:exp:comp}

			% Experiment 1 was Standard First
			% Experiment 2 was NE First
			% Talk about the comparisons
			% Talk about how the data MEANs something
			% Participants from exp1 never felt that NE was less immersive, either as immersive or more - quote feedback
			% Participants from exp2 varied a lot more, sometimes NE was more immersive, sometimes  the same, sometimes less so.

			\begin{figure}[H]
				\includegraphics[width=0.7\textwidth]{Images/Compare_Immersion_Exp_1}
				\centering
				\caption{Comparison of participants sense of immersion in the two test scenes, from Experiment 1}
				\label{exp:fig:compare_immersion_exp1}
			\end{figure}

			\begin{figure}[H]
				\includegraphics[width=0.7\textwidth]{Images/Compare_Immersion_Exp_2}
				\centering
				\caption{Comparison of participants sense of immersion in the two test scenes, from Experiment 2}
				\label{exp:fig:compare_immersion_exp2}
			\end{figure}

			\begin{figure}[H]
				\includegraphics[width=0.7\textwidth]{Images/Compare_Immersion_Variation}
				\centering
				\caption{Mean values, and variation of Range for Immersion in the two test scenes}
				\label{exp:fig:compare_immersion_variation}
			\end{figure}

	\section{Summary}
	\label{exp:summary}
		% TODO - Here I will be evaluating the results of the experiments, covering any trends that appeared between the various experiments, discussing potential impacts from the results, as well as covering any additional notes that were provided about the experiments from the participants which weren't directly related to the specific experiment they were part of.

		% Results show that an NE environment can be more immersive than a standard env, so long as the user has had a chance to familiarise themselves in a standard euclid setting first
		% More work needs doing to improve upon the navigation side, there were mixed reviews from both scenes and experiments - quote feedback - room for further study

	
	% Project Evaluation
	\chapter{Project Evaluation}
\label{eval}
\begin{multicols*}{2}

	% This is the one bit where you can talk in the first person, so go wild.

\end{multicols*}

	% Conclusion
	\chapter{Conclusion}
\label{conclusion}

	The aim of this project was to gain an insight into the effects of non-Euclidean geometry on a users sense of immersion in virtual environments, more specifically when in Virtual Reality.
	To do this, test scenes were created using both Euclidean and non-Euclidean geometry, which would be shown to users as a way to gather feedback on various metrics surrounding their sense of immersion in the scenes. % TODO: Reword this? make it longer?

	Results from the experiments conducted indicate that immersion is possible in non-Euclidean virtual environments.
	As well as this, the data from the experiments show that a user's sense of immersion can be increased compared to a standard Euclidean environment, if they transition from the standard environment to a non-Euclidean one.

	The outcomes of the experiments show that video games which make use of VR systems could additionally benefit from the use of non-Euclidean space, as a way to both expand the features of the game and increase user immersion at the same time.

	The results from the experiments open up room for future research into the subject area, such as the effects different control methods like a gamepad or room-scale motion tracking have on a users sense of immersion in a non-Euclidean environment, or perhaps how different uses of non-Euclidean space individually effect a users sense of immersion.


	% Bibliography
	\nocite{*}	% Make all entries in the bibliography file appear in the references section, even if not cited directly
	\begingroup
		\chapter{References}
		\label{ref}

		\renewcommand{\chapter}[2]{}		% Change chapter behaviour to make it not force the bibliography onto the following page
		\bibliographystyle{apalike}			% Set the bibliography to use APA formatting
		\bibliography{FinalYearProject}	    % Select the file to use for the bibliography
	\endgroup

	% Appendices
	\chapter{Appendix}
\label{appendix}

	% List of Figures
	\begingroup
		\section{List of Figures}
		\label{appendix:figures}

		Below are a list of the figures which are present in the document, along with their corresponding page number.

		\renewcommand{\chapter}[2]{}		% Change chapter behaviour to make it not force the list of figures onto the following page
		\listoffigures						% Display the list of figures
	\endgroup

	\section{Code Samples}
	\label{appendix:code}

		Below are code snippets for sections or complete functions that have specifically been referenced in the body of the report.

		\begin{lstlisting}[caption="Camera Positioning - CameraRenderPosition.cs", label=appendix:code:camera]
Vector3 offset = PointOfView.position - _player.position + ((_player.position - _player.parent.parent.position) / 2);

// Position and rotate the cameras depending on the type of illusion they are going for
if (!Inverse) {
	Moveable.position = Helper.RotatePointAroundPivot(RenderPosition.position - offset, RenderPosition.position, _relativePortalRot.eulerAngles);
	Moveable.rotation = _relativePortalRot * Quaternion.Euler(rotationOffset + _defaultRot - _normalisedDefaultRot);
} else {
	Moveable.position = RenderPosition.position - offset;
	Moveable.rotation = _relativePortalRot * Quaternion.Euler((RenderPosition.transform.up == Vector3.up ? rotationOffset : -rotationOffset) + _defaultRot - _normalisedDefaultRot + new Vector3(0, 180f, 0));
}

// Set the near clipping plane of the camera to only render starting from the closest visible area
_cam.nearClipPlane = (Helper.FindClosestPoint(_bounds, transform.position) - transform.position).magnitude / 2f;
		\end{lstlisting}

		\begin{lstlisting}[caption="Player Positioning - CameraRenderPosition.cs", label=appendix:code:player]
/// <summary>
/// Re-Position the player from their current position to the equivelant position of its point of view, depending on their relative position
/// </summary>
/// <param name="player">Player to potentially transport</param>
/// <param name="centre">Centre of the currently colliding object</param>
/// <param name="forward">Forward vector of the currently colliding object</param>
public void positionPlayer (Collider player, Vector3 centre, Vector3 forward) {
	// Get the distance vector between the player and the centre of the currently colliding object
	Vector3 distance = player.transform.position - centre;
	Quaternion inverseFlip = Quaternion.Euler(0, 0, 0);

	// If only one of the points are inverted, make sure to flip the player
	if (Inverse != _linkedScript.Inverse) {
		inverseFlip = Quaternion.Euler(0, 180f, 0);
	}

	// If the player is more than half way through the object, transport them to the linked area
	if (Vector3.Dot(distance.normalized, forward) < 0) {
		// Set player position
		distance = Helper.RotatePointAroundPivot(distance, Vector3.zero, (_relativePlayerRot.eulerAngles + inverseFlip.eulerAngles));
		player.transform.position = PointOfView.position;
		player.transform.position += distance;

		// Set player rotation
		rotationOffset += _relativePlayerRot.eulerAngles + inverseFlip.eulerAngles;
		player.transform.rotation = _relativePlayerRot * inverseFlip * player.transform.rotation;

		// Update the players momentum
		_playerControl.UpdateMoveThrottle(_relativePlayerRot * inverseFlip * _playerControl.GetMoveThrottle());
	}
}
		\end{lstlisting}

	\section{Ethics Review}
	\label{appendix:ethics}
		A completed and signed copy of the Project Ethical Review Form can be found on the following two pages.

		\includepdf[pages={-},scale=0.9]{"../Ethical Review Scan"}

	\section{Questionnaire}
	\label{appendix:question}
		A blank copy of the form used to gather the feedback discussed in Chapter \ref{exp} can be seen in the following two pages. The completed versions of the forms containing the raw data gathered are available upon request. Either the 1 or 2 were highlighted under the 'Experiment' section at the top of each page before being given to a participant, as a reference to which experiment the form was for, without indicating to the participant the nature of the experiment.

		\includepdf[pages={-},scale=1]{"Resources/Project Feedback Questionnaire"}
		

% End the document
\end{document}
