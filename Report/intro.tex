\chapter{Introduction}
\label{intro}

	Virtual Reality (VR) is fast becoming a mainstream platform for video games, with high quality devices allowing the feeling of \enquote{presence} being released for computers, game consoles, and mobile devices alike, creating a vast user base for potential products.

	As well as this, the popularity of video games which utilise non-standard or real-world geometric principles are also increasing, with the popularity of titles such as Antichamber \cite{Antichamber2013} and Portal 2 \cite{Portal22011} engaging consumers with their non-conformity to the physics and geometry of the real world.

	\section{Aims and Solutions}

		The main aim of this study is to understand the effects that non-Euclidean geometry has on a user's sense of immersion when navigating a virtual environment in VR.

		To achieve this aim, a software system is to be developed which is capable of designing and creating pseudo-non-Euclidean virtual environments.
		Using this system, experiments can then be undertaken which are used to collect data regarding user's feelings of immersion when navigating within standard Euclidean, and non-Euclidean virtual environments.
		This data can then be used to get baseline understandings about how different geometries in VR effect a user's sense of immersion.
		On top of this, the data can also be used to get an understanding about whether or not transitioning between the two impacts a user's sense of immersion compared to just experiencing one.

	\section{Project Deliverable}

		The deliverable for this project is a combination of two main components:
		\begin{itemize}
			\item A system capable of designing and creating non-Euclidean virtual environments.
			The system should be able to output to a VR system, and utilise head tracking to allow users to observe the environments freely.
			As well as this, the system should allow the users seamlessly navigate the virtual environments, making use of the non-Euclidean space.

			\item A comprehensive analysis of data gathered during experiments undertaken using the created system.
			The analysis should be able to answer the initial question defined for this study: \enquote{Is Immersion Possible in Non-Euclidean Virtual Environments?}.
			The analysis should also discuss any other conclusions which can be made from the results, and outline potential areas for future research identified from the conclusions.
		\end{itemize}

	\section{Structure}

		This report is split into five sections, excluding the introduction, each covering a different stage in the development process of the project.

		\begin{itemize}
			\item Chapter 2 covers the research conducted into existing literature that is related to the project area.
			The research is analysed, reviewed, and compared against related material, which will influence the decisions made during the development process of the project deliverable.

			\item Chapter 3 follows the development process of the system created for the construction of non-Euclidean virtual environments.
			Covered in this section is the implementation of the various pieces of functionality within the created system, as well as the design process undertaken for the scenes used for the experiments.

			\item Chapter 4 covers the experiments undertaken for the study.
			Included in this is a description of the various experiment scenarios, the metrics measured during the experiments, as well as an analysis of the results and any implications the results show.

			\item Chapter 5 is a personal reflective analysis of the completed project as a whole.
			This includes an analysis of what went well in each step of the project process, what could have been done better, and whether or not the completed project successfully covers the initial aims and objectives.

			\item Finally, chapter 6 is a conclusion of the project.
			This includes a summary of the findings of the project, discussing the outcomes of the results, and any windows it opens up for further research into the subject area.
		\end{itemize}
